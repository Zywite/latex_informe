\documentclass[12pt]{article}

% ----- Paquetes -----
\usepackage[utf8]{inputenc}
\usepackage[spanish]{babel}
\usepackage{amsmath, amssymb}
\usepackage{graphicx}
\usepackage{hyperref}
\usepackage{geometry}
\geometry{a4paper, margin=2.5cm}
\setlength{\parskip}{1em}
\setlength{\parindent}{0pt}

% ----- Datos -----
\title{Informe en \LaTeX{} con Control de Versiones en GitHub}
\author{Joaquin Alfonso Carrasco Duran}
\date{\today}

\begin{document}

\maketitle
\tableofcontents
\newpage

% ----- Introducción -----
\section{Introducción}
En este informe se abordan los conceptos fundamentales de la Programación Orientada a Objetos (POO), incluyendo \textbf{herencia}, \textbf{clases abstractas}, \textbf{polimorfismo} e \textbf{interfaces}, junto con el concepto de \textbf{Method Resolution Order (MRO)} en Python. El trabajo se gestiona mediante Git y GitHub siguiendo buenas prácticas de control de versiones y convención de commits.

% ----- Herencia -----
\section{Herencia}
La herencia permite que una clase (subclase) reutilice y extienda el comportamiento de otra (superclase).  
\begin{verbatim}
class Animal:
    def hablar(self):
        print("Sonido genérico")

class Perro(Animal):
    def hablar(self):
        print("Guau!")

p = Perro()
p.hablar()  # Imprime: Guau!
\end{verbatim}

\textbf{Ventaja:} evita duplicar código y promueve la reutilización.

% ----- Clases Abstractas -----
\section{Clases Abstractas}
Una clase abstracta define una estructura común, pero deja algunos métodos sin implementar.  
\begin{verbatim}
from abc import ABC, abstractmethod

class Figura(ABC):
    @abstractmethod
    def area(self):
        pass

class Circulo(Figura):
    def __init__(self, radio):
        self.radio = radio

    def area(self):
        return 3.14 * self.radio ** 2
\end{verbatim}

\textbf{Uso:} asegurar que las subclases implementen ciertos métodos.

% ----- Polimorfismo -----
\section{Polimorfismo}
El polimorfismo permite usar un mismo método con distintos comportamientos según el tipo de objeto.  
\begin{verbatim}
def hacer_sonido(animal):
    animal.hablar()

hacer_sonido(Perro())
hacer_sonido(Gato())
\end{verbatim}

\textbf{Idea:} un mismo mensaje produce diferentes resultados.

% ----- Interfaces -----
\section{Interfaces}
En Python, las interfaces pueden simularse usando clases abstractas con solo métodos abstractos.  
\begin{verbatim}
class Enviable(ABC):
    @abstractmethod
    def enviar(self):
        pass
\end{verbatim}

Una clase que implemente esta interfaz debe definir el método \texttt{enviar()}.

% ----- Method Resolution Order (MRO) -----
\section{Method Resolution Order (MRO)}
El MRO define el orden en que Python busca métodos y atributos en las clases al usar herencia múltiple.

Ejemplo:
\begin{verbatim}
class A: pass
class B(A): pass
class C(B): pass

print(C.mro())
\end{verbatim}

\textbf{Resultado:} muestra la jerarquía lineal de búsqueda de métodos.

% ----- Conclusión -----
\section{Conclusión}
El uso de POO junto con herramientas modernas como GitHub y \LaTeX{} potencia el desarrollo estructurado, reproducible y colaborativo de proyectos grande y pequeños.

\end{document}
